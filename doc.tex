% LTeX: language=es
% !TEX program = xelatex
% !TEX options = --shell-escape -synctex=1 -interaction=nonstopmode -file-line-error "%DOC%"

\documentclass[aspectratio=169]{beamer}

\usepackage[spanish,es-noshorthands]{babel}
\usepackage{color}
\usepackage{xcolor}

\usepackage{enumerate}
\usepackage{booktabs, multirow} % for borders and merged ranges
\usepackage{soul}% for underlines
\usepackage{changepage,threeparttable}
\usepackage{float}
\usepackage{listings}
\usepackage{pdfpages}
\usepackage{minted}
\usepackage{multicol}

\usepackage{tikz}
\usetikzlibrary{shapes,arrows}

\newcount\mycount

%\graphicspath{{./}}

%\theoremstyle{definition}
%\newtheorem{definition}{Definici\'on}[section]
%\newtheorem{proposition}[definition]{Proposici\'on}
%\newtheorem{lemma}[definition]{Lema}
%\newtheorem{theorem}[definition]{Teorema}
%\newtheorem{corollary}[definition]{Corolario}
%\newtheorem{example}[definition]{Ejemplo}
%\newtheorem{observation}[definition]{Observaci\'on}
%\newtheorem{problem}{Problema}
%\newtheorem{question}[definition]{Pregunta}
\newtheorem{pointt}{}

\def\proof{\noindent{\textbf{Demostraci\'on}}\\}
\def\endproof{\hfill{\ensuremath\square}}
\def\refname{Referencias}
\allowdisplaybreaks



\usetheme{Boadilla}
\setbeamertemplate{blocks}[rounded][shadow=false]

\usefonttheme[onlymath]{serif}

\RequirePackage{fontspec}
%\setmainfont[Ligatures=TeX]{EB Garamond}
\setsansfont[Ligatures=TeX]{EBGaramond-VariableFont_wght.ttf}
%\setseriffont[Ligatures=TeX]{Raleway}
\newfontfamily\raleway{RalewayThin-wght350.ttf}
\newfontfamily\ebg{EBGaramond-VariableFont_wght.ttf}
%\newfontfamily{\semibold}{RalewayThin-Weight600}

\setmonofont{FiraCode-Regular.ttf}
%\usepackage{mathspec}
%\setmathrm{FiraCode-Regular.ttf}
%\setmathfont(Digits,Latin){FiraCode-Regular.ttf}
%\setmathfont[range=\mathit]{FiraCode-Regular.ttf}

\definecolor{backgroundColour}{RGB}{29, 29, 38}
\definecolor{textColour}{RGB}{179, 179, 212}
\definecolor{structureColour}{RGB}{255, 51, 153}
\definecolor{structure2Colour}{RGB}{204, 102, 255}
\definecolor{structure3Colour}{RGB}{255, 204, 102}
\setbeamercolor{normal text}{fg=textColour,bg=backgroundColour}
\setbeamercolor{structure}{fg=structureColour}

\setbeamercolor{palette primary}{use=structure,bg=structureColour, fg=textColour!50!white}
\setbeamercolor{palette secondary}{use=structure,bg=structureColour, fg=textColour!50!white}
\setbeamercolor{palette tertiary}{use=structure,bg=structureColour, fg=textColour!50!white}

\setbeamercolor{author}{fg=structure2Colour}
\setbeamercolor{subtitle}{fg=structure3Colour}
\setbeamercolor{alerted text}{fg=structure3Colour}
\setbeamercolor{highlighted}{fg=structure3Colour}

%\setbeamerfont{normal text}{family=\ebg}
%\setbeamerfont{structure}{family=\ebg}
%\setbeamerfont{block body}{family=\ebg}
\setbeamerfont{title}{size=\fontsize{18}{1},family=\raleway}
\setbeamerfont{frametitle}{size=\fontsize{18}{28},family=\raleway}



\title{Sistema experto}

\author[Castro-Sotelo, García-Espinoza, Molina-Rebolledo] % (optional)
{C.A.~Castro-Sotelo \and A.T.~García-Espinoza \and I.~Molina-Rebolledo}

\institute[BUAP] % (optional)
{
  Facultad de Ciencias de la Computación\\
  Benemérita Universidad Autónoma de Puebla
}

\date[Otoño 2022] % (optional)
{Inteligencia Artificial, otoño 2022}



\date{\today}
\def\code#1{\mintinline[fontsize=\small]{lean}{#1}}
\beamerdefaultoverlayspecification{<+->}

\begin{document}
\frame{\titlepage}
%Highlighting text


\begin{frame}
\frametitle{Descripción del sistema experto}

\begin{itemize}[<+->]
\item El sistema experto es un programa que utiliza conocimiento especializado para resolver problemas.
\item Dada la escasez de agua que han sufrido algunos países en los últimos años, se ha creado un sistema experto que permite a los usuarios saber que recomendaciones de alimentos requieren de menos agua.
\item Nuestra propuesta es un sistema experto que recomienda alimentos que sean eficientes en el uso del agua.
\end{itemize}
\end{frame}

\begin{frame}
\frametitle{¿Cómo funciona nuestro sistema experto?}

\begin{itemize}[<+->]
\item El sistema experto se basa en el conocimiento de la investugación que se ha realizado sobre el tema.
\item Se muestra una interfaz gráfica que permite al usuario elegir la categoría de alimentos que desea consultar.
\item Al elegir una categoría, se muestra una lista de alimentos que pertenecen a esa categoría.
\item Al elegir un alimento, se muestra la cantidad de agua que se necesita para producirlo, así como la
  información nutricional del mismo.
\end{itemize}
\end{frame}

\begin{frame}
\frametitle{El sistema de recomendación}

\begin{itemize}[<+->]
  \item Al visualizar la información de un alimento, se muestra un botón que permite al usuario
    visualizar una lista de alimentos que requieren de menos agua para producirse.
  \item En esta lista podemos hacer clic en un alimento para ver su información en detalle.
  \item Si el usuario desea regresar a la lista de alimentos de la categoría, puede hacer clic en el botón
    "Regresar".
  \item Este menú de navegación es generado dinámicamente para cada alimento que se visualiza.
  \item Para lograr esto nos apoyamos en las funciones lógicas de Prolog.
\end{itemize}
\end{frame}

\begin{frame}[fragile]
\frametitle{Definición del sistema de recomendación}
Nuestro sistema de recomendación se basa en la siguiente regla:
%minted
\begin{block}{Código Prolog}
\begin{minted}{prolog}
recomendacion(X, Y) :-
  comida(X,A1,nutricion(F1,CH1,P1,C1),_,_,_),
  comida(Y,A2,nutricion(F2,CH2,P2,C2),_,_,_),
  A2 < A1,
  similar(F1, F2, 5), similar(CH1, CH2, 5), 
  similar(P1, P2, 5), similar(C1, C2, 15).
\end{minted}
\end{block}
\end{frame}

\end{document}
